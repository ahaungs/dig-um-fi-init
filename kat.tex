\section{KAT: Astroparticle Physics} 

\sectauthor{John Bulava, DESY}


The data produced in astroparticle physics observations is stored and 
distributed in different ways. We here collect information from several 
experiments to which scientists at DESY-Zeuthen belong. 

For some established Gamma-ray observatories such as H.E.S.S. and
VERITAS, data is stored conventionally at a few sites and is accessed
and transferred conventionally using standard protocols/tools such as
ssh, ftp, and Globus. While flexible and easy to implement and extend,
this paradigm does not efficiently accommodate growing numbers of users,
and relies on each user having explicit access to the site in question. 

Other observatories, such as CTA (planned) and IceCube, make heavy use
of the (US-based) OpenScienceGrid, as well as ad-hoc federated computing
contributions from member institutions. OpenScienceGrid is based on
GlideinWMS, which in turn uses HTCondor. Ad-hoc federation of smaller
and heterogenous computing clusters at member institutions is provided
by \url{https://github.com/WIPACrepo/pyglidein}, which also uses the
HTCondor glidein mechanism to provision slots to a central pool.
Software and other shared data resources are provisioned mainly through
CVMFS. Data movement is handled by custom middleware using GridFTP as a
transport.

This federated model has many advantages, not least of which is an
improvement in data transfer and computational throughput. Nonetheless,
some users have difficulty navigating a heterogenous pool and adapting
their workflows to different environments. Debug cycles can be extremely
long, as there is often no direct line of contact between the person
responsible for the workflow and the remote site administrator. Making
effective use of opportunistic computing resources requires active and
engaged site administrators, as well as clear service level agreements.

Another (minor) disadvantage with the grid model is the cumbersome user
identification process to obtain grid certificates. For the CTAO, the
particular grid implementation and access policy is as yet undecided, but
four sites (of which DESY is one) have been designated as CTAO data
centers. 
