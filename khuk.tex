\section{KHuK: Hadron and Nuclear Physics}

\sectauthor{Kilian Schwarz, GSI}

\subsection{Existing Federated Infrastructures}
In particular the large KHuK experiments ALICE at the Large Hadron Collider (LHC) at CERN
and the FAIR\todo{gloss! (as opposed to Findable,...)} experiments at GSI in Darmstadt have experienced a dramatic inflation of both the amount of
data and the required resources to process them. 
Both the ALICE experiment during LHC Run 3 and the CBM experiment at FAIR will take data in the order of magnitude of TB/s.
Online event selection will be done via large compute facilities on site, e.g. the Green IT Cube at GSI,
using a number of cores in the order of magnitude of 100.000 complemented by GPUs for real time reconstruction
so that in the end data of the order of magnitude of 100 PB will be stored to disk.

ALICE data are being distributed and processed by using the WLCG infrastructure. 
The Worldwide LHC Computing Grid (WLCG) project has
been founded to handle and process the data produced by the LHC
experiments. Still today  the WLCG is the world's largest computing
grid. It consists of around 170 computing centers in more than 40
countries.  It is supported by many associated national and
international grids across the world, such as European Grid
Infrastructure (EGI) and Open Science Grid (OSG), based in the US, as well as many other
regional grids.  The WLCG consists of four layers, or ``tiers''; 0, 1, 2
and 3\todo{It would be great if the grid-using communities would discuss
that once and then refer to that one discussion}. Each tier provides a specific set of services. The Tier 0 is at
CERN and is responsible for the safe-keeping of the raw data (first
copy). The large Tier 1 centers are responsible for a proportional
share of raw and reconstructed data, large-scale reprocessing and
safe-keeping of corresponding output, as well as distribution of data to Tier
2s. The Tier 2s are typically universities and other scientific
institutes, which can store sufficient data and provide adequate
computing power for specific analysis tasks. They handle analysis
requirements and proportional share of simulated event production and
reconstruction.



\subsection{Technologies Employed}
Currently the WLCG defines four component layers,  networking, hardware, middleware and
physics analysis software. The most important middleware stacks used
in the WLCG are from the European Middleware Initiative, which
combines several middleware providers (ARC, gLite, UNICORE and
dCache); the Globus Toolkit developed by the Globus Alliance; and the
Virtual Data Toolkit. The ALICE experiment is using the middleware
AliEn\todo{Are the different ``middlewares'' (what do they do, btw?) an
accident or by design?}.
The storage middleware which is used by ALICE for data storage and data transfer is XRootD.
This is to some extend also in use by GSI/FAIR experiments.
Analysis frameworks of the large KHuK experiments are usually based on the CERN analysis framework ROOT.
ROOT provides also the file structure in which the data are being stored.


\subsection{Current Issues, future challenges}
In the realm of LHC computing, the data volume and complexity will
increase dramatically with the upcoming high-luminosity phase of the
collider. Compared to 2016, an increase of computing resources by a
factor of 60 will be needed. However, assuming a flat budget, the
increase through improved technologies is expected to be only 6 to
10.\todo{merge with KET?} 
For ALICE a significant increase of data volume and complexitiy is already starting now, with
LHC Run 3.
Also the
experiments at FAIR  and and new large scale projects in astro-particle
physics will have similar requirements. 

To meet these challenges, new technologies for the access and use of
existing and future computing resources will be needed. The new
technologies evolve around the concept of a ``science cloud'' with the
approach to implement complete infrastructures as services (IaaS,
Infrastructure as a Service). This way, not only individual
applications, but also complex infrastructures consisting of a
multitude of services can be provided virtually. 

One central point is to optimise the computing models and infrastructures in place.
The idea is to concentrate data storage on a few large centres interconnected with high speed bandwidth.
This central data cloud is to be accessible in a transparent and efficient way by heterogenous compute infrastructures 
consisting of classic pleged Grid resources, but also opportunistic and heterogeneous resources as HPC centres and Cloud infrastructures,
which is where the virtualisation techniques mentioned above need to come into place.
In addition dynamic disk caches need to be provided in order to provide efficient data access from opportunistic resources and also
in order to reduce operation cost since simple caches require less maintenance work than full pledged storage elements.
Also efficient resource usage would increas by providing central and larger compute and storage facilites which can be jointly used by
several scientific communities.

Moreover due to the possibilities provided by modern bandwidths and access protocols the tasks being applied
to individual tier centres in WLCG become more complex and flexible.

One way ALICE and FAIR are addressing the challanges provided by the necessity to process huge amounts of data
in short time is to process a significant amount of these data already at the online compute facilities O2 and the Green IT Cube on site.
In addition to that ALICE is providing dedicated Analysis Facilities to the community which are optimised for high speed data processing,
such as the GSI Analysis facility.

In order to make efficient use of accelerators as GPUs the experiment code needs to be adapted.

One challenge also for the large KHuK experiments is to do data management and data analysis following the {\em FAIR} principles.
Open data, open science, and community overarching data analysis is a central challenge which needs to be addressed.

Another central challenge for KHuK is the support of the many small experiments and communities.
They are by far not as far as the larger experiments in using federated computing and storage infrastructures,
applying {\em FAIR principles} and also in corresponding software development.

\subsection{Infrastructures to be Federated \label{lab:ketfed} }
The resources to be used consist of installations with dedicated CPU
and storage systems at large computing centers, but also analysis
farms at smaller institutes as well as only temporarily available
resources at partner instutions and HPC centers (including
supercomputing resources of the Gauss-alliance). But also commercial
cloud systems and cloud access at universities and research facilities
need to be included. The use of GPU clusters is an attractive option
that is investigated as well. For the efficient and robust operation
of experiment specific software at such a variety of heterogeneous
resources a high level of abstraction of work flows is required and
the development of dedicated software tools is needed. This is needed
to allow the use of a multitude of resources spread across multiple
locations without the requirement for special expertise by the local user
community. The use of more federated and commonly used resources will
also allow to use them in a more flexible way to increase the
utilization and occupancy of resources.\todo{Anything on archiving?  Or
is that always zenodo anyway and that's enough?}

\subsection{First approaches for solutions, future technologies}
Since 2017 the German ErUM community (Erforschung von Universum
und\todo{Merge with ket?}
Materie) has started to work on common solutions to the challenges
listed above. The pilot project IDT-UM (2017-2021) is being continued
with the FIDIUM project (2021-2024), in which the intelligent
techniques for abstraction, data management and accounting will be
developed. IDT-UM started developing overlay batch systems via which dynamic science clouds
can be brought into existence. In addition to that dynamic disk caching systems have been
developed so that a first step for efficient data analysis via dynamic science clouds has been done.
These works are to be continued in the FIDIUM project. Central tasks here are
ongoing development of the overlay batch system including data aware scheduling, to provide 
compute sites in a box with minimum administrative overhead as well as accounting and monitoring tools.
A second field of development are real time monitoring systems for data lakes,
efficient integration of dynamic disk caches and usage of parallel ad hoc file systems as caches in HPC centres.
This will be complemented by data replication and placement methods for data lakes and usage
driven data management and access methods. Based on these developments prototype data lakes will be provided.
Another central point is tests, documentation and generalisation of production and analysis environments.

% editing.
